\newabbreviation{ph}{ph}{praticien hospitalier}
\newabbreviation{ch}{ch}{centre hospitalier}
\newabbreviation{naco}{naco}{Nouveaux anticoagulants oraux}
\newabbreviation{adl}{adl}{Score ADL ( Activities of Daily Living) de \textsc{Katz}}
%
\newabbreviation{anova}{anova}{analysis of variance}
\newabbreviation{aic}{aic}{Akaike information criterion}
\newabbreviation{acg}{acg}{Artérite à cellules géantes}
\newabbreviation{dgno}{dgno}{Diamètre de la Gaine du Nerf Optique}
\newabbreviation{icc}{icc}{Coefficient de corrélation intraclasse }
% \newabbreviation{}{}{}



\newglossaryentry{alpha}{name={Risque $\alpha$}, description={Probabilité de rejeter à tort l'hypothèse nulle càd conclure à une différence alors qu'il n'y en a pas.}}

\newglossaryentry{puissance}{name={Puissance}, description={1-$\beta$, $\beta$ étant la probabilité de rejeter l'hypothèse nulle quand elle est fausse càd conclure à l'absence de différence alors qu'elle existe.}}

\newglossaryentry{auc}{name={\textsc{auc}}, description={Area Uder Curve. Aire sous la courbe ROC. Un résultat à 0.5 représente un test non discriminant, 1 un test parfait. On considère habituellement qu'un résutat au dessus de 0,8 est marqueur d'un test utilisable en clinique.}}

\newglossaryentry{imput}{name={imputation}, plural
={imputations}, description={Processus de remplacement des données manquantes par des valeurs calculées. Avec les techniques actuelles d'imputation multiple par équations enchaînées on estime que les résultats sont acceptables jusqu'à environ \qty{20}{\percent} de données manquantes dans la variable.}}


